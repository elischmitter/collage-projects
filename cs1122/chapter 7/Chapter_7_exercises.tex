\documentclass{article}
\usepackage{amssymb}
\usepackage{amsmath}
\usepackage{tikz}
\usepackage{pgfplots}

\usepackage{listings}
\usepackage{color}

\definecolor{codegreen}{rgb}{0,0.6,0}
\definecolor{codegray}{rgb}{0.5,0.5,0.5}
\definecolor{codepurple}{rgb}{0.58,0,0.82}
\definecolor{backcolour}{rgb}{0.95,0.95,0.92}

\lstdefinestyle{mystyle}{
   backgroundcolor=\color{backcolour},
   commentstyle=\color{codegreen},
   keywordstyle=\color{magenta},
   numberstyle=\tiny\color{codegray},
   stringstyle=\color{codepurple},
   basicstyle=\footnotesize,
   breakatwhitespace=false,
   breaklines=true,
   captionpos=b,
   keepspaces=true,
   numbers=left,
   numbersep=5pt,
   showspaces=false,
   showstringspaces=false,
   showtabs=false,
   tabsize=2
}

\lstset{style=mystyle}

\begin{document}

\begin{flushleft}
  Eli Schmitter
\end{flushleft}
\section{R7.1}
If you try to open an file for reading that doesn't exist a FileNotFoundException will be thrown. If you try to open an file for writing that doesn't exist a new file length of 0 will be made.
\section{R7.2}
If you try to open read only file an exception will be thrown.
\section{R7.7}
A checked exception is an exception that is checked for at compile time. An example of this would be an IOException.
 An unchecked exception is an exception that is not checked at compile time. An example of this would be ArithmeticException.
\section{R7.10}
If there is no matching catch clause at a higher level, then the program will end showing stack trace of the exception. If its at a lower level it pushes there error up a stack frame
\section{R7.12}
The exception object is always the same as the type declared in the catch clause.
\section{E7.4}
\begin{lstlisting}[language=Java]
import java.io.*;
import java.util.Scanner;
import java.util.*;
public class lineNums {
    public static void main(String args[]) {
        Scanner userIn = new Scanner(System.in);
        System.out.print("Enter input file name\n>");
        String path = userIn.nextLine();
        try {
            File fileIn = new File(path);
            Scanner scnr = new Scanner(fileIn);
            System.out.print("Enter output file name\n>");
            String fileOut = userIn.nextLine();
            File fileOutPath = new File(fileOut);
            userIn.close();
            PrintWriter out = new PrintWriter(fileOutPath);
            int lineNumber = 1;
            while (scnr.hasNextLine()) {
                String line = scnr.nextLine();
                out.println("\\* " + lineNumber + " *\\ " + line);
                lineNumber++;

            }
            out.close();
            scnr.close();
        } catch (FileNotFoundException e) {
            System.out.print("File not found");
        }

    }

}
\end{lstlisting}
\end{document}
