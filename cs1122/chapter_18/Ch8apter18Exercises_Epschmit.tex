\documentclass{article}
\usepackage{amssymb}
\usepackage{amsmath}

\usepackage{tikz}
\usepackage{listings}
\usepackage{color}
\usepackage{enumitem}
\usepackage{listings}
\usepackage{color}

\definecolor{codegreen}{rgb}{0,0.6,0}
\definecolor{codegray}{rgb}{0.5,0.5,0.5}
\definecolor{codepurple}{rgb}{0.58,0,0.82}
\definecolor{backcolour}{rgb}{0.95,0.95,0.92}

\lstdefinestyle{mystyle}{
    commentstyle=\color{codegreen},
    keywordstyle=\color{magenta},
    numberstyle=\tiny\color{codegray},
    stringstyle=\color{codepurple},
    basicstyle=\footnotesize,
    breakatwhitespace=false,
    breaklines=true,
    captionpos=b,
    keepspaces=true,
    numbers=left,
    numbersep=5pt,
    showspaces=false,
    showstringspaces=false,
    showtabs=false,
    tabsize=2
}
\newlist{alphalist}{enumerate}{1}
\setlist[alphalist,1]{label=\textbf{\Alph*.}}

\begin{document}
\begin{flushleft}
 Eli Schmitter\\
\end{flushleft}
\section{R18.1}
A type parameter is a parameter That gives the type of a generic class is using in some way. An example of an type parameter is the underlined section of ArrayList\textless\underline{String}\textgreater example=new ArrayList\textless\textgreater(); The second block of <> is implicitly  assigned to string as well so a type parameter is not needed
\section{R18.2}
The difference between a generic class and a regular class is that a generic class takes a type parameter and a regular class does not. Generic class are useful when dealing with data structure's so you don't have to write a  regular class for each possible object that needs to be stored.
\section{R18.6}
\begin{enumerate}
  \item Hashmap
  \item Treemap
  \item Hashtable
  \item LinkedHasMap
\end{enumerate}
\section{E18.21}
\begin{lstlisting}[language=Java]
public static <E> boolean  isPalindrome(ArrayList<E> x) {
 for(int i =0;i<x.size()-1;i++ ){
     if(!(x.get(i).equals(x.get(x.size()-1-i)))){
         return false;
     }
 }
 return true;
 }

 \end{lstlisting}
\end{document}
