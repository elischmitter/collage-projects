\documentclass{article}
\usepackage{amssymb}
\usepackage{amsmath}


\usepackage{listings}
\usepackage{color}
\usepackage{enumitem}
\usepackage{listings}
\usepackage{color}

\definecolor{codegreen}{rgb}{0,0.6,0}
\definecolor{codegray}{rgb}{0.5,0.5,0.5}
\definecolor{codepurple}{rgb}{0.58,0,0.82}
\definecolor{backcolour}{rgb}{0.95,0.95,0.92}

\lstdefinestyle{mystyle}{
    commentstyle=\color{codegreen},
    keywordstyle=\color{magenta},
    numberstyle=\tiny\color{codegray},
    stringstyle=\color{codepurple},
    basicstyle=\footnotesize,
    breakatwhitespace=false,
    breaklines=true,
    captionpos=b,
    keepspaces=true,
    numbers=left,
    numbersep=5pt,
    showspaces=false,
    showstringspaces=false,
    showtabs=false,
    tabsize=2
}

\usepackage{pdfpages}
\newlist{alphalist}{enumerate}{1}
\setlist[alphalist,1]{label=\textbf{\Alph*.}}
\begin{document}

\begin{flushleft}
  Eli Schmitter\\
  \today
\end{flushleft}
\section{R13.1}
\begin{alphalist}
\item Recursion is where a function calls it self.
\item Iteration is where an iterate with a loop
\item Infinite Recursion is where there is no special case and it keeps calling it self with out returning anything.
\item Helper Methods are Methods that are used to make a one parameter to two or more.
\end{alphalist}
\section{R13.4}
\begin{lstlisting}
public class Sort{
  public int[] sort(int[] arr){
    return bubble(arr,arr.length);
  }
  public int[] bubble(int[] arr,int n){
      ...
      if (n-1>1){
      return bubble(arr n-1);
      }
  }
}
\end{lstlisting}
\section{R13.9}
\begin{lstlisting}[language=Java]
public class Fib {
    static int fibcount=0;
    public static int fib(int n)
    {
        fibcount+=1;
        if (n <= 1) return n;
        else return fib(n-1) + fib(n-2);
    }

    public static void main(String[] args) {
        for(int i=1;i<1000;i++) {
            fibcount=0;
            System.out.println(i+"::"+fib(i) + "::" + fibcount);
        }
    }
}
\end{lstlisting}
the first cuple of out puts are:\\
fib index :: fib number :: fibcount\\
1::1::1\\
2::1::3\\
3::2::5\\
4::3::9\\
5::5::15\\
6::8::25\\
7::13::41\\
8::21::67\\
9::34::109\\
10::55::177\\
...\\
37::24157817::78176337\\
38::39088169::126491971\\
39::63245986::204668309\\
40::102334155::331160281\\
41::165580141::535828591\\
42::267914296::866988873\\\\
 the later numbers it looks like the operations approached $2^n$ where n is the fib index
\section{R13.5}
\begin{lstlisting}[language=Java]
public static String flip(String s)
{
    if(s.length()==0) {
        return "";
    }
    return flip(s.substring(1))+s.charAt(0);
}
\end{lstlisting}


\end{document}
