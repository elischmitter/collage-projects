\documentclass{article}
\usepackage{amssymb}
\usepackage{amsmath}

\usepackage{tikz}
\usepackage{listings}
\usepackage{color}
\usepackage{enumitem}
\usepackage{listings}
\usepackage{color}

\definecolor{codegreen}{rgb}{0,0.6,0}
\definecolor{codegray}{rgb}{0.5,0.5,0.5}
\definecolor{codepurple}{rgb}{0.58,0,0.82}
\definecolor{backcolour}{rgb}{0.95,0.95,0.92}

\lstdefinestyle{mystyle}{
    commentstyle=\color{codegreen},
    keywordstyle=\color{magenta},
    numberstyle=\tiny\color{codegray},
    stringstyle=\color{codepurple},
    basicstyle=\footnotesize,
    breakatwhitespace=false,
    breaklines=true,
    captionpos=b,
    keepspaces=true,
    numbers=left,
    numbersep=5pt,
    showspaces=false,
    showstringspaces=false,
    showtabs=false,
    tabsize=2
}
\newlist{alphalist}{enumerate}{1}
\setlist[alphalist,1]{label=\textbf{\Alph*.}}

\begin{document}
\begin{flushleft}
 Eli Schmitter\\
 \today
\end{flushleft}
\section{R15.1}
The implantation of the invoices should be a list. The major differences between a list and a set is that a set is un ordered and has no duplicates. The order does not matter for the invoice but the not having duplicates does. The point of an invoices is to have know the total cost for the purchased item, and thanks to the associative property of addition $1+2=2+1$ but we need duplicates if there where multiples of an item purchased
\section{R15.3}
You can use a map that uses a data for its key and stores an array list of the events in the day.
\section{R15.8}
\begin{enumerate}
 \item staff:
 \item staff:\\
       iterator:
 \item staff: Tom\\
       iterator:\underline{Tom}
 \item staff: Tom Diana\\
       iterator:\underline{Diana}
 \item staff: Tom Diana Harry\\
       iterator:\underline{Harry}
 \item staff: Tom Diana Harry\\
       iterator:
 \item staff: Diana Harry\\
       iterator:{Tom}
 \item
       \begin{enumerate}
        \item staff: Diana Harry\\
        iterator:{Diana}
        \item staff: Diana Harry\\
        iterator:{Harry}
       \end{enumerate}
\end{enumerate}

\section{R15.16}
This is assuming that the letters are pushed on to the first stack in the order A,B,C,D...Z
\begin{itemize}
  \item A
  \item B
  \item C
  \item D\\
  .\\
  .\\
  .
  \item Z
\end{itemize}
\section{R15.21}
A map can be thought as a set of (key, value) pairs. This can be done because of the order does not matter and there can not be two of the same keys.
\section{R15.21}
\begin{lstlisting}[language=Java]

import java.util.Stack;
import java.util.Scanner;

class Test
{
    public static void stackafy(int x){
        Stack<Integer>s=new Stack<>();
        while(x!=0){
            s.push(x%10);
            x=(int)Math.floor(x/10);
        }
        while(s.size()!=0){
            System.out.print(s.pop()+" ");
        }
    }
    public static void main(String args[]) {
        Scanner reader = new Scanner(System.in);
        System.out.println("Enter a number: ");
        int n = reader.nextInt();
        stackafy(n);
    }
}


\end{lstlisting}
\end{document}
