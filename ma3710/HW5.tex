\documentclass[12pt]{report}
\usepackage[margin=1in]{geometry}
\usepackage[english]{babel} 
\usepackage{amsmath,amsfonts,amsthm}
\usepackage{fancyhdr}
\usepackage{enumitem}

\usepackage{fourier}
\usepackage{mathtools}

\pagestyle{fancy}
\fancyhf{}
\rhead{Eli Schmitter}
\lhead{HW5}
\rfoot{\thepage}

\begin{document}
{\bf 2}\\

\begin{enumerate}[label={\bf \alph*}]
\item 
\begin{align*}
P(X\leq 2) &= P(X =0)+P(X=1) +P(X=2)\\
&= .4 + .3 +.12\\
&=.85
\end{align*}
\item 
\begin{align*}
P(X >1) &= P(X=2)+P(X=3) +P(X=4) + P(X=5)\\
&= .15 + .1 +.05\\
&=.3
\end{align*}
\item
\begin{align*}
\mu_X &= \sum_x xP(X = x)\\
&= 0(P(X=0))+ 1(P(X=1))+ 2(P(X=2))+ 3(P(X=3))+ 4(P(X=4))\\
&= 0(.4)+ 1(.3)+ 2(.15)+ 3(.1)+ 4(.05)\\
&=1.1
\end{align*}
\item
\begin{align*}
\delta^2_X &= \sum_X(X^2)P(X=x) - \mu_X^2\\
&=(0^2(P(X=0))+ 1^2(P(X=1))+ 2^2(P(X=2))+ 3^2(P(X=3))+ 4^2(P(X=4))-1.1^2\\
&=(0^2(.4)+ 1^2(.3)+ 2^2(.15)+ 3^2(.1)+ 4^2(.05)-1.1^2\\
&=1.39
\end{align*}
\end{enumerate}

{\bf 4 a}\\

\par probability mass function must sum up to 1 that is $\sum_X P(X=x) =1$\\
\begin{align*}
\sum_X P_1(X=x) &=.2+.2+.2+.3+.1+.1=.9\\
\sum_X P_2(X=x) &=.1+.3+.3+.2+.2=1.1\\
\sum_X P_3(X=x) &=.1+.2+.4+.2+.1=1\\
\end{align*}
\par There for $P_3$ could be a  possible  probability  mass  function  of X.

{\bf 8}
\begin{enumerate}[label={\bf \alph*}]
\item 
\begin{align*}
F(x)&=P(X \leq x) \\
P(X \leq 2) &= F(2)\\
&=.83
\end{align*}
\item 
\begin{align*}
P(X > 3)&=1-P(X \leq 3)\\
&= 1-F(3)\\
&=.05
\end{align*}
\item 
\begin{align*}
P(X =1)&=F(1)-F(0)\\
&= .72 -.41 \\
&=.31\\
\end{align*}
\item
\begin{align*}
P(X =0)&=P( X \leq 0)\\
&=F(0)\\
&=.41
\end{align*}
\item
\begin{align*}
P(X =0)&=F(0) = .41\\
P(X=1)&=F(1)-F(0)=.31\\
P(X=2)&=F(2)-F(1)=.11\\
P(X=3)&=F(3)-F(2)=.12\\
P(X=4)&=F(4)-F(3)=.05\\
\end{align*}

\par $P(X=0)$ has the highest probability. 
\end{enumerate}

{\bf 24}%a-e
\begin{enumerate}[label={\bf \alph*}]
\item 
\begin{align*}
\int_{-\infty}^\infty f(t) \text{dt} &= 1 \\
1&=0+\int_1^\infty \frac{C}{x^3} \text{dx}\\
&=\frac{C}{2x^2} \bigg\rvert_1^{\infty}\\
&=\frac{C}{2(\infty)^2}-\frac{C}{-2(1)^2}\\
C&=2
\end{align*}
\item
$$\mu_X= \int_{-\infty}^{\infty} x f(x)\text{dx}$$
\begin{align*}
\mu_X &= 0+\int_{1}^{\infty} x \frac{2}{x^3} \text{dx}\\
      &= 2
\end{align*}
\item
\begin{align*}
F(x) &= \int_{-\infty}^x f(t) \text{dt}\\
&=\int_{1}^x \frac{2}{t^3} \text{dt}\\
&=1-\frac{1}{x^2}\text{ for} x\geq 1  
\end{align*}
for $x < 1, F(x) =0$
$$ F(x) = 
   \begin{cases*}
      0 &x<0\\
      1-\frac{1}{x^2} & x \geq 1
   \end{cases*}
   $$
\item
\begin{align*}
.5 &= \int_{-\infty}^{x_m} f(X) \text{dx}\\
   &=  \int_{-\infty}^{x_m} \frac{2}{x^3} \text{dx}\\
   &= 1-\frac{1}{x_m^2}\\
1.412&=x_m
\end{align*}
\item
\begin{align*}
\text{PM}_{10}=P(X \leq 10) &= F(10)\\
&=1-\frac{1}{(10)^2}\\
\text{PM}_{10}&=.99
\end{align*}
\end{enumerate}
 
\end{document}