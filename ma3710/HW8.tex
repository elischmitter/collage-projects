\documentclass[12pt]{report}
\usepackage[margin=1in]{geometry}
\usepackage[english]{babel} 
\usepackage{amsmath,amsfonts,amsthm}
\usepackage{fancyhdr}
\usepackage{enumitem}
\usepackage{indentfirst}

\pagestyle{fancy}
\fancyhf{}
\rhead{Eli Schmitter}
\lhead{HW6}
\rfoot{\thepage}

\begin{document}
\section*{4.7}
   \subsection*{4}
      \begin{enumerate}[label={\bf \alph*}]
      \item
         \begin{align*}
                  \mu_X&=\frac{1}{\lambda}\\
                     12&=\frac{1}{\lambda}\\
               \lambda &=\frac{1}{12}\\
         \end{align*}
      \item
         \begin{align*}
                      P(X\leq m)&=.5\\
            1-e^{-\frac{1}{12}m}&=.5\\\
                              m &=-12\text{ln}(.5)\\
                                &= 8.318\\
         \end{align*}
         \item 
            \begin{align*}
               \sigma^2_X &= \frac{1}{\lambda^2}\\
                          &= 144\\
               \sigma_X &= 12
            \end{align*}
         \item
          \begin{align*}
                      P(X\leq p_{65})&=.65\\
            1-e^{-\frac{1}{12} p_{65}}&=.65\\
                        \lambda &=\frac{1}{12}\\
                              m &=-12\text{ln}(.35)\\
                                &= 12.59
         \end{align*}
      \end{enumerate}
\section*{4.11}
   \subsection*{2}
         N(.08, .0001)
   \begin{enumerate}[label={\bf \alph*}]
      \item

      \begin{align*}
         \frac{20.2}{.2}     &\leq X\\
         \frac{X-\mu}{\sigma}&=Z\\
                             &\Rightarrow\\
                       .7881 &\leq Z\\
                       P(X>20.2)&=1-.7881=.2119
      \end{align*}
      \item
      \begin{align*}
         10\%&\Rightarrow -1.28=Z\\
         -1.28*\sigma+\mu&=.0672\\
         .0672*250&=16.8\\
         &\Rightarrow\\
         16.8 \text{mm thick}
      \end{align*}
      \item
      Yes due to the Central Limit Theorem, one can use normal distribution.
      \begin{align*}
       z&=\frac{.1-.08}{.01}\\
        &=2\\
        &\Rightarrow\\
        P(X>.1)1-.9772=.0228
      \end{align*}
   \end{enumerate}
   \subsection*{6}
   \begin{enumerate}[label={\bf \alph*}]
   \item
   \begin{align*}
    \bar{X}~N(\mu,\frac{\sigma^2}{n}) &\vee P(\bar{X}>1.305)\\
                                      &\Rightarrow\\
                              \sigma_X&=\frac{\sigma^2}{n}\\
                                      &=\frac{.1}{200^.5}\\
                                      &=.00707\\
                                     Z&=\frac{X-\mu_x}{\sigma_X}\\
                                      &=\frac{1.305-1.3}{.00707}\\
                                      &=.707\\
                      P(\bar{X}>1.305)&=  1-.7794\\
                                      &= .2206
   \end{align*}
   \item
   \begin{align*}
         -.68&=\frac{X_{25}-\mu_x}{.00707}\\
       X_{25}&= 1.29 \text{mm}
   \end{align*}
   \item
   \begin{align*}
      .05 &\Rightarrow Z=1.65\\
      1.65&=\frac{1.305-1.3}{\frac{.1^2}{n}}\\
          n &= 1083
   \end{align*}
   \end{enumerate}
   \subsection*{10}
      \begin{align*}
          \mu&=np\\
             &=100*.3\\
             &=30\\
       \sigma&=\sqrt{\mu(1-p)}\\
             &=\sqrt{30*(1-.3)}\\
             &=4.58\\
      P(X>35)&=P(Z>\frac{35-30}{4.58})\\
             &=P(Z>1.2)\\
             &=1-.8849\\
             &=.1151
      \end{align*}
\end{document}