\documentclass[12pt]{article}
\usepackage{times}
\usepackage{fullpage}
\usepackage{amsfonts}
\usepackage[T1]{fontenc} % Added to be able to use \textquotedbl
                         % \textquotedbl doesn't work in math mode

\setlength{\parindent}{0in}
\setlength{\parskip}{0.1in}

\begin{document}


% http://tex.stackexchange.com/questions/116779/aligning-using-flushleft-and-flushright
\begin{center} % just for vertical spacing and killing indent
\begin{tabular*}{\textwidth}{@{}l@{\extracolsep{\fill}}r@{}}
{\large CS3311 Homework 1}
  & Due: Wednesday, September 11, 2019, 8:59am \\
  & Submission: Typed, pdf on Canvas \\
  & (scanned submissions are not allowed)
\end{tabular*}
\end{center}
\vspace{-0.2in}
\rule{\textwidth}{0.5mm}
\begin{small}
The answers must be the original work of the author.  While discussion
with others is permitted and encouraged, the final work should be done
individually. You are not allowed to work in groups.  You are allowed to
build on the material supplied in the class. Any other source must be
specified clearly.
\end{small}
\rule{\textwidth}{0.5mm}

Consider the following sets for questions {\bf 1} and {\bf 2}:
\begin{center}
$X = \{a, 2, \{a\}, [a], [a,a], [\emptyset, a] \}$ \hspace{0.10in} 
\hspace{0.1in}
$Y = \{a, 3,        [a], [a,a], \emptyset, [a,\emptyset] \}$ 
\end{center}

{\bf 1.} {\em (20 points)} Write out each of the sets listed below.


  {\bf (a) } $\; X \cup Y$\\
$\{a,2,\{a\},[a],[a,a],[\emptyset,a],3,\emptyset,[a,\emptyset]\}$\\


  {\bf (b) } $\; X \cap Y$\\
$\{a,[a],[a,a]\} $\\


  {\bf (c) } $\; X - Y$\\
$\{2,\{a\},[\emptyset,a]\}$\\\

  {\bf (d) } $\; P(X - \{ [a,a], [\emptyset, a]\})$\\
$\{\{\},\{a\},\{2\},\{\{a\}\},\{[a]\},\{a,2\},\{a,\{a\}\},\{a,[a]\},\{2,\{a\}\},\{2,[a]\},\{\{a\},[a]\},\{a,2,\{a\}\},\{a,2,[a]\}\\\{a,\{a\},[a]\},\{2,\{a\},[a]\},\{a,2,\{a\},[a]\}\}$\\


  {\bf (e) } $\; \{a,1\} \times \{ a, [a], \{a\}, \emptyset \} $ \\
$ \{[a,a],[a,[a]],[a,\{a\}],[a,\emptyset],[1,a],[1,[a]],[1,\{a\}],[1,\emptyset]\} $

\vspace{0.1in}

{\bf 2.} {\em (20 points)} State whether the following propositions are
TRUE or FALSE.


  {\bf (a) } $\; a \in X$\\
True\\
  {\bf (b) } $\; \{a\} \in X$\\
True\\
  {\bf (c) } $\; a \in Y$\\
True\\
  {\bf (d) } $\; \{a\} \in Y$\\
False\\
  {\bf (e) } $\; \emptyset \in X$\\
False\\
  {\bf (f) } $\; \emptyset \in Y$\\
True\\
  {\bf (g) } $\; \emptyset \subseteq Y$\\
True\\
  {\bf (h) } $\; \{ \emptyset \} \subseteq X$\\
False\\
  {\bf (f) } $\; \{ \emptyset \} \subseteq Y$\\
True\\
  {\bf (j) } $\; \{[a,a]\} \in X \times X$ \\
True\\

\hfill {\em Please turn the page over for additional questions. }
\newpage

{\bf 3.} {\em (20 points)} 

{\bf (a) }
Write the first 5 elements of the set $S_1$ defined recursively.
Put the basis elements as the first members.
Assume that the arithmetic computations defined in the recursive step 
will be performed to obtain the new elements of $S_1$.

\begin{quote}
{\bf (i) Basis:} $[1,1] \in S_1$

{\bf (ii) Recursive step:} If $[n,m] \in S_1$, then
$[n+1, m+ 2(n+1)- 1 ] \in S_1$.

{\bf (iii) Closure:} $S_1$ consists of exactly the elements that can be
obtained by starting with the basis elements of $S_1$ and applying the
recursive step finitely many times to construct new elements of $S_1$.
\end{quote}

$S_1= \{[1,1],[2,4],[3,9],[4,16],[5,25]\}$\\


{\bf (b) }
Write the first 6 elements of the set $S_2$ defined recursively.
Put the basis elements as the first members.
The first member of the basis sequence is a number, and the second
member is a string. The recursive step performs an arithmetic addition on the 
first member and string concatenation on the second member.
The symbols $a, b$ are characters, not variables.
\begin{quote}
{\bf (i) Basis:} $[1,a] \in S_2$ and $[1,b] \in S_2$

{\bf (ii) Recursive step:} If $[n,w] \in S_2$, then 
$[n+2, awa] \in S_2$ and $[n+2, bwb] \in S_2$.

{\bf (iii) Closure:} $S_2$ consists of exactly the elements that can be
obtained by starting with the basis elements of $S_2$ and applying the
recursive step finitely many times to construct new elements of $S_2$.
\end{quote}

$S_2=\{[1,a],[1,b],[3,aaa],[3.bab],[3,aba],[3,bbb]\}$

\vspace{0.1in}

{\bf 4.} {\em (5+15 points)}

Consider the following infinite set $A$.

$A = \{ 1, 5, 13, 29, 61, \ldots \} $

{\bf (a) }
Describe the pattern to obtain an element from the previous element.\\

{\bf (b) }
 Give a recursive definition of the set $A$.\\
{\bf (i) Basis:} $1 \in A$\\
{\bf (ii) Recursive step:} If $n \in A$, then $(n* 2)+3\in A$.\\
{\bf (iii) Closure:} $A$ consists of exactly the elements that can be
obtained by starting with the basis elements of $A$ and applying the
recursive step finitely many times to construct new elements of $A$.

\vspace{0.1in}

{\bf 5.} {\em (5+15 points)}

Consider the following infinite set $B$.

$B = \{ a, babb, bbabbbb, bbbabbbbbb, \ldots \} $

{\bf (a) }
Describe the pattern to obtain an element from the previous element.\\
if $w \in B$ then $bwbb \in B$\\

{\bf (b) }
 Give a recursive definition of the set $B$.\\
\begin{quote}
{\bf (i) Basis:} $a \in B$\\
{\bf (ii) Recursive step:} If $w\in B$, then $bwbb \in B $.

{\bf (iii) Closure:} $B$ consists of exactly the elements that can be
obtained by starting with the basis elements of $B$ and applying the
recursive step finitely many times to construct new elements of $B$.
\end{quote}
\end{document}


